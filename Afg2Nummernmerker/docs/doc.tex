\documentclass[12pt]{article}

\usepackage[margin=3.5cm]{geometry}

\usepackage[ngerman]{babel}
\usepackage[T1]{fontenc}
\usepackage[utf8]{inputenc}
%\usepackage{babel}

\usepackage{color}
\usepackage{listings}
\usepackage{courier}

%\setmonofont{Consolas} %to be used with XeLaTeX or LuaLaTeX
\definecolor{bluekeywords}{rgb}{0,0,1}
\definecolor{greencomments}{rgb}{0,0.5,0}
\definecolor{redstrings}{rgb}{0.64,0.08,0.08}
\definecolor{xmlcomments}{rgb}{0.5,0.5,0.5}
\definecolor{types}{rgb}{0.17,0.57,0.68}

\usepackage{listings}
\lstset{language=[Sharp]C,
captionpos=b,
%numbers=left, %Nummerierung
%numberstyle=\tiny, % kleine Zeilennummern
frame=lines, % Oberhalb und unterhalb des Listings ist eine Linie
showspaces=false,
showtabs=false,
breaklines=true,
showstringspaces=false,
breakatwhitespace=true,
escapeinside={(*@}{@*)},
commentstyle=\color{greencomments},
morekeywords={partial, var, value, get, set},
keywordstyle=\color{bluekeywords},
stringstyle=\color{redstrings},
basicstyle=\ttfamily\small,
literate=%
    {Ö}{{\"O}}1
    {Ä}{{\"A}}1
    {Ü}{{\"U}}1
    {ß}{{\ss}}1
    {ü}{{\"u}}1
    {ä}{{\"a}}1
    {ö}{{\"o}}1
    {~}{{\textasciitilde}}1
}

\begin{document}

\title{Aufgabe 2 - Nummernmerker}

\section{Lösungsidee}

Um die optimale Aufteilung zu ermitteln, verwende ich eine Variation des Knapsack-Algorithmus. Dieser funktioniert wie folgt:

\begin{lstlisting}
TeileNummerAuf(nullstellen) {
    if ([Bereits für gleiche Parameter aufgerufen])
        return [Bereits errechnetes Ergebnis];

    for (int i in 2..4) {
        subAufteilung = TeileNummerAuf(nullstellen[(i+1)..]);

        Möglichkeiten.Add(subAufteilung.PrecedeBy(i));
    }

    return Möglichkeiten.Max(aufteilung => BewerteAufteilung(nullstellen, aufteilung));
}

BewerteAufteilung(nullstellen, aufteilung) {
    return [Anzahl an führenden nullstellen in der Aufteilung];
}
\end{lstlisting}

\end{document}