\documentclass[a4paper,10pt,ngerman]{scrartcl}
\usepackage{babel}
\usepackage[T1]{fontenc}
\usepackage[utf8x]{inputenc}
\usepackage[a4paper,margin=2.5cm,footskip=0.5cm]{geometry}

% Die nächsten drei Felder bitte anpassen:
\newcommand{\Aufgabe}{Aufgabe 4: Urlaubsfahrt} % Aufgabennummer und Aufgabennamen angeben
\newcommand{\TeamID}{00587}       % Team-ID aus dem PMS angeben
\newcommand{\TeamName}{Doge.NET} % Team-Namen angeben
\newcommand{\Namen}{Johannes von Stoephasius} % Namen der Bearbeiter/-innen dieser Aufgabe angeben
 
% Kopf- und Fußzeilen
\usepackage{scrlayer-scrpage, lastpage}
\setkomafont{pageheadfoot}{\large\textrm}
\lohead{\Aufgabe}
\rohead{Team-ID: \TeamID}
\cfoot*{\thepage{}/\pageref{LastPage}}

% Position des Titels
\usepackage{titling}
\setlength{\droptitle}{-1.0cm}

% Für mathematische Befehle und Symbole
\usepackage{amsmath}
\usepackage{amssymb}

% Für Bilder
\usepackage{graphicx}

% Für Algorithmen
\usepackage{algpseudocode}

% Für Quelltext
\usepackage{listings}
\usepackage{color}
\definecolor{mygreen}{rgb}{0,0.6,0}
\definecolor{mygray}{rgb}{0.5,0.5,0.5}
\definecolor{mymauve}{rgb}{0.58,0,0.82}
\lstset{
  keywordstyle=\color{blue},commentstyle=\color{mygreen},
  stringstyle=\color{mymauve},rulecolor=\color{black},
  basicstyle=\footnotesize\ttfamily,numberstyle=\tiny\color{mygray},
  captionpos=b, % sets the caption-position to bottom
  keepspaces=true, % keeps spaces in text
  numbers=left, numbersep=5pt, showspaces=false,showstringspaces=true,
  showtabs=false, stepnumber=2, tabsize=2, title=\lstname
}
\lstdefinelanguage{JavaScript}{ % JavaScript ist als einzige Sprache noch nicht vordefiniert
  keywords={break, case, catch, continue, debugger, default, delete, do, else, finally, for, function, if, in, instanceof, new, return, switch, this, throw, try, typeof, var, void, while, with},
  morecomment=[l]{//},
  morecomment=[s]{/*}{*/},
  morestring=[b]',
  morestring=[b]",
  sensitive=true
}

% Diese beiden Pakete müssen zuletzt geladen werden
%\usepackage{hyperref} % Anklickbare Links im Dokument
\usepackage{cleveref}

% Daten für die Titelseite
\title{\textbf{\Huge\Aufgabe}}
\author{\LARGE Team-ID: \LARGE \TeamID \\\\
	    \LARGE Team-Name: \LARGE \TeamName \\\\
	    \LARGE Bearbeiter dieser Aufgabe: \\ 
	    \LARGE \Namen\\\\}
\date{\LARGE\today}

\begin{document}

\maketitle
\tableofcontents

\vspace{0.5cm}

\section{Lösungsidee}
Die Grundidee der Lösung basiert auf der Tatsache, dass jede ideale Lösung in 2 kleinere jeweils auch ideal lösbare Teilprobleme heruntergebrochen werden kann. In diesem Fall wird für jede Tankstelle iterativ basierend auf den bisher berechneten Wegen der beste Gefunden, wobei die Stationen der Position nach aufsteigend sortiert sind. Auf diese Art und Weise kann jedes Mal die ideale Lösung bestimmt werden. \\
Die ideale Betankungsmenge für jedes Teilproblem wird errechnet, indem mit dem Preis aufsteigend immer so voll wie es geht getankt wird, wie es geht, es sei denn, dass breits durch eine preiswertere Tankstelle der Bereich betankt ist. 

\section{Umsetzung}
Der Hauptalgorithmus liegt in der Urlaubsfahrt.cs in der Urlaubsfahrt-Klasse in der Funktion GetTack, die als Argumente die Streckenlänge, das Start-Benzin, die Strecke, die mit einer maximalen Tankfüllung zurückgelegt werden kann und eine Liste aller Tankstellen. Tankstellen sind in GasStation.cs in der Klasse GasStation dargestellt und enthalten den Preis und die Position. Die Klasse Track aus Track.cs  enthält eine Liste an Tankstellen und berechnet für diese Zusammensetzung an Tankstellen den besten Preis, egal an wie vielen Tankstellen gehalten wird.
Die Hauptberechnung in der GetTrack-Funktion berechnet für jede Tankstelle den besten Weg zu ihr, indem aus jedem bereits berechneten Track ein neuer mit dem aktuellen erstellt wird. Der beste wird der Liste der bereits berechneten Tracks hinzugefügt und die nächste Iteration beginnt.

\section{Beispiele}
Genügend Beispiele einbinden! Die Beispiele von der BwInf-Webseite sollten hier diskutiert werden, aber auch eigene Beispiele sind sehr gut – besonders wenn sie Spezialfälle abdecken. Aber bitte nicht 30 Seiten Programmausgabe hier einfügen!

\section{Quellcode}
Unwichtige Teile des Programms sollen hier nicht abgedruckt werden. Dieser Teil sollte nicht mehr als 2–3 Seiten umfassen, maximal 10.


\end{document}

