\documentclass[a4paper,10pt,ngerman]{scrartcl}
\usepackage{babel}
\usepackage[T1]{fontenc}
\usepackage[utf8x]{inputenc}
\usepackage[a4paper,margin=2.5cm,footskip=0.5cm]{geometry}

% Die nächsten drei Felder bitte anpassen:
\newcommand{\Aufgabe}{Aufgabe 4: Urlaubsfahrt} % Aufgabennummer und Aufgabennamen angeben
\newcommand{\TeamID}{00587}       % Team-ID aus dem PMS angeben
\newcommand{\TeamName}{Doge.NET} % Team-Namen angeben
\newcommand{\Namen}{Johannes von Stoephasius} % Namen der Bearbeiter/-innen dieser Aufgabe angeben
 
% Kopf- und Fußzeilen
\usepackage{scrlayer-scrpage, lastpage}
\setkomafont{pageheadfoot}{\large\textrm}
\lohead{\Aufgabe}
\rohead{Team-ID: \TeamID}
\cfoot*{\thepage{}/\pageref{LastPage}}

% Position des Titels
\usepackage{titling}
\setlength{\droptitle}{-1.0cm}

% Für mathematische Befehle und Symbole
\usepackage{amsmath}
\usepackage{amssymb}

% Für Bilder
\usepackage{graphicx}

% Für Algorithmen
\usepackage{algpseudocode}

% Für Quelltext
\usepackage{listings}
\usepackage{color}
\definecolor{mygreen}{rgb}{0,0.6,0}
\definecolor{mygray}{rgb}{0.5,0.5,0.5}
\definecolor{mymauve}{rgb}{0.58,0,0.82}
\lstset{
  keywordstyle=\color{blue},commentstyle=\color{mygreen},
  stringstyle=\color{mymauve},rulecolor=\color{black},
  basicstyle=\footnotesize\ttfamily,numberstyle=\tiny\color{mygray},
  captionpos=b, % sets the caption-position to bottom
  keepspaces=true, % keeps spaces in text
  numbers=left, numbersep=5pt, showspaces=false,showstringspaces=true,
  showtabs=false, stepnumber=2, tabsize=2, title=\lstname
}
\lstdefinelanguage{JavaScript}{ % JavaScript ist als einzige Sprache noch nicht vordefiniert
  keywords={break, case, catch, continue, debugger, default, delete, do, else, finally, for, function, if, in, instanceof, new, return, switch, this, throw, try, typeof, var, void, while, with},
  morecomment=[l]{//},
  morecomment=[s]{/*}{*/},
  morestring=[b]',
  morestring=[b]",
  sensitive=true
}

% Diese beiden Pakete müssen zuletzt geladen werden
%\usepackage{hyperref} % Anklickbare Links im Dokument
\usepackage{cleveref}

% Daten für die Titelseite
\title{\textbf{\Huge\Aufgabe}}
\author{\LARGE Team-ID: \LARGE \TeamID \\\\
	    \LARGE Team-Name: \LARGE \TeamName \\\\
	    \LARGE Bearbeiter dieser Aufgabe: \\ 
	    \LARGE \Namen\\\\}
\date{\LARGE\today}

\begin{document}

\maketitle
\tableofcontents

\vspace{0.5cm}

\section{Lösungsidee}
Die Hauptfunktion enthält eine Liste, TrackParts, die für jede Tankstelle den besten Track zu ihr enthält. Dieser wird gebildet, indem für jede Tankstelle davor der beste Track gefunden wird. Aus all diesen bereits berechneten Tracks wird jeweils ein neuer gebildet, indem von jedem zuerst die mit den wenigsten Stopps ausgewählt werden und dann die, die am günstigsten sind, falls es mehrere sind. Die Berechnung des Preises eines Abschnitts wird im nächsten Abschnitt erklärt. Dieser Lösungsweg findet das richtige Ergebnis, da eine ideale Lösung immer in 2 selbst auch wieder ideal lösbare Teilprobleme zerlegt werden kann. \\
Der beste Preis eines Wegabschnitts wird mit Hilfe von Abdeckung gelöst. Dafür werden die Tankstellen dem Preis entlang aufsteigend geordnet. Zuerst wird von der preiswertesten Tankstelle aus eine Range definiert, die von der Tankstelle aus bis zur maximalen Tanklänge, oder wenn die Gesamt-Streckenlänge kürzer ist, bis zum Streckenende geht. Danach wird für die zweit-preiswerteste Tankstelle auch eine solche Range definiert, es sei denn, es gibt Überschneidungen mit einer bereits gefundenen Range. In diesem Fall wird die bereits gefundene Range erweitert. Dieses Verfahren wird so lange wiederholt, bis nur noch eine Range existiert, die den gesamten Weg abdeckt. Dieses Verfahren liefert das ideale Ergebnis, da preislich aufsteigend immer die beste Teillösung gefunden wird. 

\section{Umsetzung}
\subsection{Codestruktur}
\begin{enumerate}
	\item Urlaubsfahrt.cs
	\begin{enumerate}
		\item enthält die statische Urlaubsfahrt, die die statische Methode GetTrack enthält
		\item GetTrack erhält als Parameter die Streckenlänge, das Start-Benzin, die Strecke, die mit einer maximalen Tankfüllung zurückgelegt werden kann und eine Liste aller Tankstellen
	\end{enumerate}
\end{enumerate}
\subsubsection{Implementierung}
\section{Beispiele}
Genügend Beispiele einbinden! Die Beispiele von der BwInf-Webseite sollten hier diskutiert werden, aber auch eigene Beispiele sind sehr gut – besonders wenn sie Spezialfälle abdecken. Aber bitte nicht 30 Seiten Programmausgabe hier einfügen!

\section{Quellcode}
Unwichtige Teile des Programms sollen hier nicht abgedruckt werden. Dieser Teil sollte nicht mehr als 2–3 Seiten umfassen, maximal 10.


\end{document}

