\documentclass[a4paper,10pt,ngerman]{scrartcl}
\usepackage{babel}
\usepackage[T1]{fontenc}
\usepackage[utf8x]{inputenc}
\usepackage[a4paper,margin=2.5cm,footskip=0.5cm]{geometry}

% Die nächsten drei Felder bitte anpassen:
\newcommand{\Aufgabe}{Aufgabe 3: Telepaartie} % Aufgabennummer und Aufgabennamen angeben
\newcommand{\TeamID}{00587}       % Team-ID aus dem PMS angeben
\newcommand{\TeamName}{Doge.NET} % Team-Namen angeben
\newcommand{\Namen}{Johannes von Stoephasius \& Nikolas Kilian} % Namen der Bearbeiter/-innen dieser Aufgabe angeben
 
% Fonts
\usepackage{lmodern}
\usepackage{inconsolata}
%\usepackage{courier}

% Kopf- und Fußzeilen
\usepackage{scrlayer-scrpage, lastpage}
\setkomafont{pageheadfoot}{\large\textrm}
\lohead{\Aufgabe}
\rohead{Team-ID: \TeamID}
\cfoot*{\thepage{}/\pageref{LastPage}}

% Position des Titels
\usepackage{titling}
\setlength{\droptitle}{-1.0cm}

% Für mathematische Befehle und Symbole
\usepackage{amsmath}
\usepackage{amssymb}

% Für Bilder
\usepackage{graphicx}

% Für Algorithmen
\usepackage{algpseudocode}

% Für Quelltext
\usepackage{listings}
\usepackage{xcolor}

%\setmonofont{Consolas} %to be used with XeLaTeX or LuaLaTeX
\definecolor{bluekeywords}{rgb}{0,0,1}
\definecolor{greencomments}{rgb}{0,0.5,0}
\definecolor{redstrings}{rgb}{0.64,0.08,0.08}
\definecolor{xmlcomments}{rgb}{0.5,0.5,0.5}
\definecolor{types}{rgb}{0.17,0.57,0.68}
\definecolor{background}{rgb}{0.95,0.95,0.95}

\lstdefinelanguage{CSharp}{ % Better C# highlighting
language=[Sharp]C,
backgroundcolor=\color{background},
captionpos=b,
numbers=left, %Nummerierung
numberstyle=\tiny, % kleine Zeilennummern
frame=lrtb,
showspaces=false,
showtabs=false,
breaklines=true,
showstringspaces=false,
breakatwhitespace=true,
escapeinside={(*@}{@*)},
commentstyle=\color{greencomments},
morekeywords={partial, var, value, get, set},
keywordstyle=\color{bluekeywords},
stringstyle=\color{redstrings},
basicstyle=\ttfamily\small,
literate=%
    {Ö}{{\"O}}1
    {Ä}{{\"A}}1
    {Ü}{{\"U}}1
    {ß}{{\ss}}1
    {ü}{{\"u}}1
    {ä}{{\"a}}1
    {ö}{{\"o}}1
    {~}{{\textasciitilde}}1
}

% Diese beiden Pakete müssen zuletzt geladen werden
\usepackage{hyperref} % Anklickbare Links im Dokument
\usepackage{cleveref}

\lstMakeShortInline[
  language=CSharp,
  columns=fixed,
  basicstyle=\ttfamily
  ,columns=fixed]|

\lstnewenvironment{lstcs}
    {\lstset{
        language=CSharp,
        basicstyle=\ttfamily,
        breaklines=true,
        columns=fullflexible
    }}
{}

% Daten für die Titelseite
\title{\textbf{\Huge\Aufgabe}}
\author{\LARGE Team-ID: \LARGE \TeamID \\\\
        \LARGE Team-Name: \LARGE \TeamName \\\\
        \LARGE Bearbeiter dieser Aufgabe: \\ 
        \LARGE \Namen\\\\}
\date{\LARGE\today}

\begin{document}

\maketitle
\tableofcontents

\vspace{0.5cm}

\section{Lösungsidee}

\subsection{Definitionen}

\paragraph{Zustand} \label{def:state}

Ein Zustand ist definiert als Menge von Behältern, wobei jedem Behälter eine nichtnegative ganze Zahl zugeordnet werden kann, die der Anzahl an Bibern im Gefäß entspricht.

Weiter können die Behälter untereinander getauscht werden, da die Konstellation die selbe bleibt. Deshalb werden die Biber-Anzahlen eines Zustands immer nur im sortierten Zustand betrachtet, wobei hier aufsteigende Sortierung verwendet wird.

Man bezeichne die Menge aller Zustände mit Gesamtbiberzahl \(n\) als \(\mathbb{S}_n\).

\paragraph{Endzustand} \label{def:target}

Ein Endzustand ist jeder Zustand, der genau einen leeren Eimer enthält.

Sind weniger, also keine, leeren Eimer enthalten, so ist der Zustand kein Endzustand laut der Aufgabenstellung.

Sind mehr enthalten, so ist der Zustand nur durch Operationen auf einen anderen Endzustand zu erhalten, und somit nicht relevant. Die Ermittlung aller Endzustände wird in \cref{alg:target} erläutert.

Man bezeichne die Menge aller Endzustände mit Gesamtbiberzahl \(n\) als \(\mathbb{E}_n\).
Dabei gilt: \(\mathbb{E}_n\subseteq\mathbb{S}_n\)

\paragraph{Ursprungszustand} \label{def:origin}

Ein Ursprungszustand von einem Zustand \(x\) ist jeder Zustand, der mit einem einzelnen Telepaartieschritt zum Zustand \(x\) wird. Die Ermittlung aller Ursprungszustände eines Zustands wird in \cref{alg:origin} erläutert.

Die Menge an Ursprungszuständen von \(x\) kann geschrieben werden als \(origin(x)\), mit \(origin : \mathbb{S}_n \to \mathbb{S}_n \).

\paragraph{Generation} \label{def:generation}

Eine Generation ist eine Menge an unterschiedlichen Zuständen.

\subsection{Kernidee}

Die Grundidee der Lösung basiert nicht auf der Idee, alle Nicht-Endzustände zu ermitteln und diese optimal zu lösen, sonder alle Endzustände zu ermitteln und diese invers auf alle ihre Ursprungszustände zurückzuführen, diese Ursprungszustände wieder auf ihre eigenen Ursprungszustände zurückzuführen usw., wobei konstant überprüft wird, ob es nicht ''Abkürzungen'' im Sinne bereits gefundener Zustände gibt.

\subsection{Ermittlung aller Ursprungszustände} \label{alg:origin}

Zum Ermitteln der Ursprungszustände \(origin(s)\) eines Zustands \(s \in \mathbb{S}_n\) wird jede Biber-Anzahl mit jeder anderen Biber-Anzahl verglichen. Ist dabei bei einem Vergleich zweier Anzahlen die erste Anzahl größer als 0 und durch 2 teilbar, dann ist es möglich, dass auf diese beiden Behälter eine Telepaartie angewendet wurde. Um diese umzukehren wird die Anzahl im ersten Behälter addiert und die Differenz zum 2. Behälter addiert.
Diese Überprüfung wird für alle Kombinationen zweier Biber-Anzahlen durchgeführt, bis am Ende alle Ursprungszustände gefunden wurden.

\subsubsection{Begründung}

Seien \(a_0, a_1, b_0, b_1 \in \mathbb{N}\) die zwei Biberanzahlen, wobei \(a_0\) und \(b_0\) die Anzahlen vor der Telepaartie repräsentieren, und \(a_1\) und \(b_1\) die danach. Sei weiterhin o.B.d.A. \(a_0 < b_0\).

Laut der Definition der Telepaartie gilt:

\begin{align}
    a_1 &= 2 a_0 \label{eq:a1} \\
    b_1 &= b_0 - a_0 \label{eq:b1} \\
\end{align}

Hieraus lässt sich herleiten:

\begin{center}
    \begin{minipage}{0.23\textwidth}
        \begin{align*}
            && a_1 &= 2 a_0 \cref{eq:a1} \\
            \iff&& a_0 &= \frac{a_1}{2} \\
        \end{align*}
    \end{minipage}
    \begin{minipage}{0.23\textwidth}
        \begin{align*}
            && b_1 &= b_0 - a_0 \cref{eq:b1} \\
            \iff&& b_1 &= b_0 - \frac{a_1}{2} \\
            \iff&& b_0 &= b_1 + \frac{a_1}{2} \\
        \end{align*}
    \end{minipage}
    \begin{minipage}{0.23\textwidth}
        \begin{align*}
            && a_0 &< b_0 \\
            \iff&& \frac{a_1}{2} &< b_1 + \frac{a_1}{2} \\
            \iff&& 0 &< b_1 \\
        \end{align*}
    \end{minipage}
    \begin{minipage}{0.23\textwidth}
        \begin{align*}
            && a_0 &&\in \mathbb{N} \\
            \iff&& a_0 &\mid 1 \\
            \iff&& a_0 &= 1 \cdot n \\
            \iff&& a_1 &= 2 \cdot n \\
            \iff&& a_1 &\mid 2 \\
        \end{align*}
    \end{minipage}
\end{center}

Wichtig hierbei ist \(0 < b_1 \land a_0 = \frac{a_1}{2} \land b_0 = b_1 + \frac{a_1}{2} \land a_1 \mid 2\).

\subsection{Generieren der Endzustände} \label{alg:target}

Zur effizienten Findung aller Endzustände werden nicht erst alle möglichen Endzustände mit Duplikaten generiert und am Ende die Duplikate entfernt, sondern gleich nur Zustände berechnet, die nicht wiederholt auftreten werden.

Zur Simplifizierung der Rechnung werden alle Zustände mit Behälterzahl minus eins berechnet, die keine leeren Behälter besitzen. Danach wird an jedes dieser einfach eine Null angehängt.

Der Algoritmus funktioniert, indem zuerst für einen Behälter alle möglichen Biberanzahlen ermittelt werden, für die gilt:

\begin{itemize}
    \item Es ist möglich die restlichen Biber so aufzuteilen, dass jeder Behälter genauso viele oder weniger Biber enthält wie der vorherigen.
    \item Es ist garantiert, dass die restlichen Behälter alle nicht leer sein müssen.
\end{itemize}

Um zu garantieren, dass die Behälter absteigend befüllbar sind, müssen mindestens \(\left\lceil\frac{<Anzahl\ Biber>}{<Anzahl\ Behälter>}\right\rceil\) Biber in den ersten Becher. 
Um die Nicht-Leerheit zu garantieren, müssen maximal \(<Anzahl\ Biber> - (<Anzahl\ Behälter> + 1)\) Biber in den ersten Becher. Somit kann mindestens ein Biber in jeden restlichen Behälter platziert werden.

Für den trivialen Fall, das nur ein Behälter vorhanden ist, müssen alle Biber in diesen.
Nun lassen sich alle für den ersten Becher möglichen Biberanzahlen bestimmen und für diese jeweils rekursiv alle folgenden Biberanzahlen. Hierbei ist noch zu beachten, dass noch garantiert werden muss, dass die Behälter absteigend voll sind. Dies ist trivialerweise umsetzbar, indem alle Biberanzahlen, die größer der Biberanzahlen eines vorherigen Behälters sind, eliminiert werden.

\subsection{Hauptalgorithmus}

Sei die Generation \(Gen_{i+1}\) definiert als die Menge aller unterschiedlichen Ursprungszustände aller Elemente aus \(Gen_i\), die in keiner vorherigen Generation \(Gen_k, k < i\) enthalten sind.

Sei dabei \(Gen_0\) als Spezialfall gleich der Menge aller Endzustände für Gesamtbiberzahl \(n\).

\[Gen_0 := \mathbb{E}_n\]
\[Gen_{i+1} := \left\{ s \mid \underbrace{\left(\exists_{t \in Gen_i} : s \in origin(t)\right)}_{\substack{\text{Alle Ursprungszustände}\\\text{der vorherigen Generation}}} \land \underbrace{\left(\forall_{i_0\in\mathbb{N}, i_0 \leq i}: s \notin Gen_{i_0}\right)}_{\substack{\text{Keine bereits in vorherigen}\\\text{Generationen enthaltene Zustände}}} \right\}\]

Der Algoritmus funktioniert dann, indem er nach und nach alle nicht-leeren Generationen ermittelt. Sei \(Gen_m\) die letzte nicht-leere Generation, so ist \(LLL(n) = m\).

\subsection{Beweis}

Es existiert eine letzte nicht-leere Generation \(Gen_m\).
Weiterdem gilt \(LLL(n) = m\).

\subsubsection{Hilfssatz 1: Erreichen aller lösbaren Zustände} \label{proof:completeness}

Wähle beliebig, aber fest einen Zustand \(s\in\mathbb{S}_n\).
Ist der Zustand lösbar, also durch wiederholte Telepaartie zu einem Endzustand überführbar, so gibt es eine Generation \(Gen_i\) aus \((Gen_i)_{i\in\mathbb{N}}\) mit \(s\in Gen_i\).

\paragraph{Trivialer Fall}
Gilt \(s\in\mathbb{E}_n\), so ist \(s\) lösbar mit 0 Telepaartieschritten. Da \(Gen_0 = \mathbb{E}_n\) gilt, gilt \(s \in Gen_0\).

\paragraph{Beweis durch Widerspruch}

Angenommen \(s \notin Gen_{i}\). Für alle \(i=1,2,...\).

\begin{align*}
    &&s \notin Gen_{i} &\iff \lnot\left(\left(\exists_{t \in Gen_{i-1}} : s \in origin(t)\right) \land \left(\forall_{i_0\in\mathbb{N}, i_0 < i}: s \notin Gen_{i_0}\right)\right) \\
    && &\iff \lnot\left(\exists_{t \in Gen_{i-1}} : s \in origin(t)\right) \lor \lnot\left(\forall_{i_0\in\mathbb{N}, i_0 < i}: s \notin Gen_{i_0}\right) \\
    && &\iff \left(\forall_{t \in Gen_{i-1}} : s \notin origin(t)\right) \lor \left(\exists_{i_0\in\mathbb{N}, i_0 < i}: s \in Gen_{i_0}\right)
\end{align*}

Angenommen es gilt \(\exists_{i_0\in\mathbb{N}, i_0 < i}: s \in Gen_{i_0}\). Wenn dies gilt, existiert ein \(i_0\), für welches gilt: \(s \in Gen_{i_0}\). Dann existiert eine Generation aus \((Gen_i)_{i\in\mathbb{N}}\) mit \(s \in Gen_{i_0}\).

Somit können wir das Problem durch Redefinition \(i := i_0\) reformulieren als:

\begin{align*}
    &&s \notin Gen_{i} 
    &\iff \forall_{t \in Gen_{i-1}} : s \notin origin(t)
\end{align*}

Dies ist nun zu zeigen:

\begin{align*}
    s \notin Gen_{i}
    &\iff \forall_{t \in Gen_i} : s \notin origin(t) \\
    \intertext{Bemerkung: \(origin(origin(t)) = \left\{ s \mid \exists_{u \in origin(t)} : s \in origin(u) \right\}\)}
    &\iff \forall_{t \in Gen_{i-1}} : s \notin origin(origin(t)) \\
    &\iff \forall_{t \in Gen_{i-1}} : s \notin (origin \circ origin)(t) \\
    &\iff \forall_{t \in Gen_{i-i}} : s \notin (\underbrace{origin \circ ... \circ origin}_{\text{i-mal verkettet}})(t) \\
    &\iff \forall_{t \in \mathbb{E}_n} : s \notin (\underbrace{origin \circ ... \circ origin}_{\text{i-mal verkettet}})(t) \\
\end{align*}

Damit dies gilt, müsste \(s\) für keine Anzahl \(i\) an Telepaartieschritten zu einem Endzustand kommen. Somit müsste \(s\) also unlösbar sein. \(\Box\)

\subsubsection{Korollar aus Hilfssatz 1: Maximale Mindestschrittzahl} \label{proof:maxminsteps}

Wenn \(s \in Gen_i\) gilt, dann ist \(s\) in \(i\) oder weniger Telepaartieschritten zu einem Endzustand überführbar.

\paragraph{Beweis}

Wie aus \cref{proof:completeness} hervorgeht, kann ein Zustand \(s\in\mathbb{S}_n\) nur lösbar sein bzw. eine Generation \(Gen_i\) mit \(s \in Gen_i\) existieren, wenn gilt:

\begin{align*}
    &&\forall_{t \in Gen_i} : s \notin origin(t) 
    &\iff \forall_{t \in \mathbb{E}_n} : s \notin (\underbrace{origin \circ ... \circ origin}_{\text{i-mal verkettet}})(t) \\ 
    \iff&& \forall_{t \in Gen_i} : s \in origin(t) 
    &\iff \exists_{t \in \mathbb{E}_n} : s \in (\underbrace{origin \circ ... \circ origin}_{\text{i-mal verkettet}})(t) \\ 
\end{align*}

Da laut Definition von \(origin\) die \(i\)-fache Selbstverkettung von \(origin\) alle Zustände sind, von denen aus der Parameter mit weniger als oder genau \(i\) Telepaartieschritten erreicht werden kann ist, ist der Endzustand \(t \in \mathbb{E}_n\) von \(s\) in weniger als oder genau \(i\) Schritten erreichbar. \(\Box\)

\subsubsection{Hilfssatz 2: Eindeutigkeit} \label{proof:uniqueness}

Für jeden lösbaren Zustand \(s \in \mathbb{S}_n\) gilt, dass \textit{genau ein} \(i\) existiert, für das die Generation \(Gen_i\) mit \(s \in Gen_i\) existiert.

\paragraph{Beweis}

Ist der Zustand \(s\) lösbar, so existiert laut \cref{proof:completeness} ein \(i\) mit \(s \in Gen_i\). Aufgrund der Kondition \(\forall_{i_0\in\mathbb{N}, i_0 < i}: s \notin Gen_{i_0}\) in der Definition von \(Gen_i\) gilt, dass keine Generation \(Gen_j, j < i\) aus \((Gen_i)_{i\in\mathbb{N}}\) existiert, die \(s\) enthält. Andersherum gibt es auch keine späteren Generationen \(Gen_k, k > i\) mit \(s \in Gen_k\), da für diese dann ein \(i_0 = i\) mit \(s \in Gen_{i_0}\) existieren würde, was gegen die Definition von \(Gen_i\) verstößt. \(\Box\)

\subsubsection{Hilfssatz 3: Minimalität der Schritte} \label{proof:minimality}

Für jeden lösbaren Zustand \(s \in \mathbb{S}_n\) mit \(s \in Gen_i\) gilt, dass \(i = LLL(s)\).

\paragraph{Beweis}

Der Fall \(LLL(s) > i\) wird vom Korollar \cref{proof:maxminsteps} widerlegt.
Somit wäre nur noch zu zeigen das \(LLL(s) < i\) nicht gilt.

Damit \(LLL(s) < i\) gilt, müsste es eine Schrittfolge geben, um \(s\) mit \(k<i\) Schritten in einen Endzustand zu überzuführen.

Die Generationen \(Gen_j\) mit \(0 \leq j < i\) enthalten zusammen alle Elemente von allen \(j\)-fachen Selbstverkettungen von \(origin\), also jeden Zustand, der in genau \(i-1\) oder weniger Schritten zu einem Endzustand überführbar ist. Mit der Eindeutigkeit der Generationen (siehe \cref{proof:uniqueness}) ist also \(LLL(s) < i\) und \(s \in Gen_j\) äquivalent.

Ist nun \(s \in Gen_i\), so gilt laut \cref{proof:uniqueness}, dass \(s\) in keiner anderen Generation aus \((Gen_i)_{i\in\mathbb{N}}\), also, auch keiner Generation \(Gen_j\) enthalten ist. Da \(LLL(s) < i \iff s \in Gen_j\) gilt und da \(s \notin Gen_j\) gilt, gilt auch \(\lnot(LLL(s) < i) \iff LLL(s) \geq i\).

Da \(LLL(s) \geq i\) gilt, gilt \(LLL(s) < i\) nicht. \(\Box\)

\subsubsection{Hilfssatz 4: Garantie der Leerheit} \label{proof:termination}

Die Serie \((Gen_i)_{i\in\mathbb{N}}\) ist bis inklusive zu einem Index \(m\) in keinem Element leer. Nach diesem Index ist sie in jedem Element leer.

\paragraph{Beweis}

Angenommen \(Gen_i = {}\).

\begin{align*}
    Gen_{i+1} :&= \left\{ s \mid \left(\exists_{t \in Gen_i} : s \in origin(t)\right) \land \left(\forall_{i_0\in\mathbb{N}, i_0 \leq i}: s \notin Gen_{i_0}\right) \right\} \\
    & = \{ s \mid (\underbrace{\exists_{t \in \{\}} : s \in origin(t)}_{\substack{\text{In der leeren Menge}\\\text{existieren keine Elemente,}\\\text{erst Recht keine, die}\\\text{die Kondition erfüllen}}}) \land (\forall_{i_0\in\mathbb{N}, i_0 \leq i}: s \notin Gen_{i_0}) \} \\
    & = \{\} \\
\end{align*}

Somit gilt \(Gen_i = {} \implies Gen_{i+1} = \{\}\).

Wäre vor einem Index \(m\) eine Generation leer, müssten somit auch folgende Generationen leer sein. Somit wäre \(m\) redefinierbar als der Index ab dem das erste leere Element vorkommt.

Da nur eine endliche Menge an Zuständen \(s \in \mathbb{S}_n\) existiert und da alle lösbaren Zustände, welche Teilmenge aller Zustände sind, laut \cref{proof:uniqueness} eindeutig genau einer Generation angehören, ist auch die Menge an nicht-leeren Generationen endlich.

Da endlich viele nicht-leere Generationen enthalten sein müssen und da die Serie nicht zwischendrin leere Generationen enthalten kann, muss sie alle nicht-leeren Generationen bis zu einem Index \(m\) enthalten, und alle leeren ab diesem. \(\Box\)

\subsubsection{Beweis}

Laut \cref{proof:termination} existiert eine letzte, nicht-leere Generation \(Gen_m\).
Da die letzte nicht-leere Menge existiert, ist bekannt, dass alle nicht-leeren Generation einen Index kleiner oder gleich \(m\) haben.
Laut \cref{proof:minimality} gilt für alle \(s \in Gen_i\): \(LLL(s) = i\).
Da alle Generation mit mehr als null Zuständen \(Gen_i\) einen Index \(i \leq m\) haben, ist die maximale \(LLL\) der maximale Index \(m\).
Da die maximale \(LLL\) gleich dem Index \(m\) ist, ist \(L(n) = m\).

\section{Umsetzung}

Zur Umsetzung haben wir den obigen Algorithmus in C\# 8.0 mit
.NET Core 3.0 implementiert.\\
Die Zustände werden in Form einer |public class| State gespeichert.
Die |class| beinhaltet

\begin{enumerate}
    \item |int Depthations| - 
        Eine Property zur Errechnung der Generationsnummer des Zustands.
    \item |State? Parent| - 
        Der Zustand, von dem der Zustand Ursprungszustand ist.
        Ist der Zustand ein Endzustand, so ist Parent |null|.
    \item |int[] Buckets| - Die Biberanzahlen.
\end{enumerate}

Die wichtigsten Methoden aus der |class| State sind |public IEnumerable<State> Origins()| und |private State ReverseTeelepartie(int first, int second)|, wobei |Origins()| alle Ursprungszustände des Zustands ermittelt, und |ReverseTeelepartie(int first, int second)| dabei intern einen neuen Zustand berechnet, der den Zustand vor der Telepaartie beschreibt. 

Die allgemeine Berechnung erfolgt in der |public static class| Telepartie. Hier ist die wichtigste Methode |private static int LLLCore(int numberOfCups, int numberOfItems, State? goal, Action<string>? writeLine)|, die entweder für nur einen gegebenen Fall oder für eine Anzahl von Bibern die Anzahl von nötigen Operationen berechnet.

\section{Beispiele}

Für mehr Info zu den Parametern des Programms führen sie |Telepaartie.CLI --help| aus.

Für die Verteilung 2, 4, 7 ist die Ausgabe:
\begin{lstcs}
Telepaartie.CLI -l 2,4,7 -v

Starting iteration 2
FERTIG!
Man benötigt 2 Telepaartie-Schritte
Die Berechnung dauerte 0:00 Minuten.
\end{lstcs}

Für die Verteilung 3, 5, 7 ist die Ausgabe:
\begin{lstcs}
Telepaartie.CLI -l 3,5,7 -v

Starting iteration 3
FERTIG!
Man benötigt 3 Telepaartie-Schritte
Die Berechnung dauerte 0:00 Minuten.
\end{lstcs}

Für die Verteilung 80, 64, 32 ist die Ausgabe:
\begin{lstcs}
Telepaartie.CLI -l 80,64,32 -v

Starting iteration 2
FERTIG!
Man benötigt 2 Telepaartie-Schritte
Die Berechnung dauerte 0:00 Minuten.
\end{lstcs}

\begin{lstcs}
Telepaartie.CLI -c 3 -e 10 -v

Starting iteration 3

--------------

State (Depth:2) {1;4;5}
State (Depth:1) {1;1;8}
State (Depth:0) {0;2;8}

--------------

State (Depth:2) {2;3;5}
State (Depth:1) {2;2;6}
State (Depth:0) {0;4;6}

--------------

State (Depth:2) {1;2;7}
State (Depth:1) {2;2;6}
State (Depth:0) {0;4;6}

--------------

State (Depth:2) {1;3;6}
State (Depth:1) {2;2;6}
State (Depth:0) {0;4;6}

--------------


FERTIG!
Man benötigt 3 Telepaartie-Schritte
Die Berechnung dauerte 0.072s.
\end{lstcs}

Für \(n = 100\) ist die Ausgabe für die LLL:
\begin{lstcs}
Telepaartie.CLI -c 3 -e 100 -v

Starting iteration 8

--------------

State (Depth:8) {31;32;37}
State (Depth:7) {5;31;64}
State (Depth:6) {10;31;59}
State (Depth:5) {10;28;62}
State (Depth:4) {20;28;52}
State (Depth:3) {8;40;52}
State (Depth:2) {16;32;52}
State (Depth:1) {32;32;36}
State (Depth:0) {0;36;64}

--------------

State (Depth:8) {5;32;63}
State (Depth:7) {5;31;64}
State (Depth:6) {10;31;59}
State (Depth:5) {10;28;62}
State (Depth:4) {20;28;52}
State (Depth:3) {8;40;52}
State (Depth:2) {16;32;52}
State (Depth:1) {32;32;36}
State (Depth:0) {0;36;64}

--------------


FERTIG!
Man benötigt 8 Telepaartie-Schritte
Die Berechnung dauerte 0.036s.
\end{lstcs}

Für \(n = 600\) mit 4 Behältern ist die Ausgabe für die LLL:
\begin{lstcs}
Telepaartie.CLI -c 4 -e 600 -v

Starting iteration 8

--------------

State (Depth:7) {1;117;191;291}
State (Depth:6) {2;117;191;290}
State (Depth:5) {4;117;189;290}
State (Depth:4) {4;173;189;234}
State (Depth:3) {8;173;189;230}
State (Depth:2) {8;16;230;346}
State (Depth:1) {16;16;222;346}
State (Depth:0) {0;32;222;346}

--------------

State (Depth:7) {3;109;198;290}
State (Depth:6) {6;106;198;290}
State (Depth:5) {6;184;198;212}
State (Depth:4) {12;178;198;212}
State (Depth:3) {12;20;212;356}
State (Depth:2) {8;24;212;356}
State (Depth:1) {16;16;212;356}
State (Depth:0) {0;32;212;356}

--------------

State (Depth:7) {23;94;202;281}
State (Depth:6) {23;187;188;202}
State (Depth:5) {14;23;187;376}
State (Depth:4) {9;28;187;376}
State (Depth:3) {9;56;159;376}
State (Depth:2) {9;103;112;376}
State (Depth:1) {18;103;103;376}
State (Depth:0) {0;18;206;376}

--------------
\end{lstcs}

- Einige Ergebnisse der Kürze halber ausgelassen -

% State (Depth:7) {1;82;200;317}
% State (Depth:6) {2;82;200;316}
% State (Depth:5) {4;82;200;314}
% State (Depth:4) {4;164;200;232}
% State (Depth:3) {4;68;200;328}
% State (Depth:2) {4;132;136;328}
% State (Depth:1) {4;4;264;328}
% State (Depth:0) {0;8;264;328}

% --------------

% State (Depth:7) {3;110;201;286}
% State (Depth:6) {6;107;201;286}
% State (Depth:5) {12;107;201;280}
% State (Depth:4) {12;79;107;402}
% State (Depth:3) {24;67;107;402}
% State (Depth:2) {24;40;134;402}
% State (Depth:1) {24;40;268;268}
% State (Depth:0) {0;24;40;536}

% --------------

% State (Depth:7) {29;71;89;411}
% State (Depth:6) {29;89;142;340}
% State (Depth:5) {29;89;198;284}
% State (Depth:4) {58;89;198;255}
% State (Depth:3) {31;116;198;255}
% State (Depth:2) {62;85;198;255}
% State (Depth:1) {62;170;170;198}
% State (Depth:0) {0;62;198;340}

% --------------

% State (Depth:7) {50;105;172;273}
% State (Depth:6) {100;105;172;223}
% State (Depth:5) {100;118;172;210}
% State (Depth:4) {72;118;200;210}
% State (Depth:3) {72;82;210;236}
% State (Depth:2) {82;144;164;210}
% State (Depth:1) {62;164;164;210}
% State (Depth:0) {0;62;210;328}

% --------------

% State (Depth:7) {65;90;111;334}
% State (Depth:6) {21;65;180;334}
% State (Depth:5) {21;115;130;334}
% State (Depth:4) {42;109;115;334}
% State (Depth:3) {6;42;218;334}
% State (Depth:2) {12;36;218;334}
% State (Depth:1) {24;24;218;334}
% State (Depth:0) {0;48;218;334}

% --------------

% State (Depth:7) {6;37;196;361}
% State (Depth:6) {12;31;196;361}
% State (Depth:5) {12;31;165;392}
% State (Depth:4) {24;31;165;380}
% State (Depth:3) {24;62;134;380}
% State (Depth:2) {24;72;124;380}
% State (Depth:1) {48;48;124;380}
% State (Depth:0) {0;96;124;380}

% --------------

% State (Depth:7) {33;85;103;379}
% State (Depth:6) {33;85;206;276}
% State (Depth:5) {33;121;170;276}
% State (Depth:4) {33;106;121;340}
% State (Depth:3) {33;106;219;242}
% State (Depth:2) {66;73;219;242}
% State (Depth:1) {66;146;146;242}
% State (Depth:0) {0;66;242;292}

% --------------

% State (Depth:7) {1;120;196;283}
% State (Depth:6) {2;120;196;282}
% State (Depth:5) {2;76;240;282}
% State (Depth:4) {4;76;240;280}
% State (Depth:3) {8;76;236;280}
% State (Depth:2) {16;76;228;280}
% State (Depth:1) {16;152;152;280}
% State (Depth:0) {0;16;280;304}

% --------------

% State (Depth:7) {58;71;106;365}
% State (Depth:6) {58;106;142;294}
% State (Depth:5) {58;142;188;212}
% State (Depth:4) {116;130;142;212}
% State (Depth:3) {26;130;212;232}
% State (Depth:2) {52;104;212;232}
% State (Depth:1) {104;104;160;232}
% State (Depth:0) {0;160;208;232}

% --------------

% State (Depth:7) {9;86;114;391}
% State (Depth:6) {18;77;114;391}
% State (Depth:5) {18;77;228;277}
% State (Depth:4) {18;49;77;456}
% State (Depth:3) {36;49;59;456}
% State (Depth:2) {13;59;72;456}
% State (Depth:1) {26;59;59;456}
% State (Depth:0) {0;26;118;456}

% --------------

% State (Depth:7) {47;53;133;367}
% State (Depth:6) {53;86;94;367}
% State (Depth:5) {53;86;188;273}
% State (Depth:4) {86;106;135;273}
% State (Depth:3) {86;135;167;212}
% State (Depth:2) {45;86;135;334}
% State (Depth:1) {86;90;90;334}
% State (Depth:0) {0;86;180;334}

% --------------

% State (Depth:7) {17;98;194;291}
% State (Depth:6) {34;98;177;291}
% State (Depth:5) {34;98;114;354}
% State (Depth:4) {68;98;114;320}
% State (Depth:3) {98;114;136;252}
% State (Depth:2) {38;114;196;252}
% State (Depth:1) {76;76;196;252}
% State (Depth:0) {0;152;196;252}

% --------------

% State (Depth:7) {9;96;194;301}
% State (Depth:6) {9;96;107;388}
% State (Depth:5) {18;87;107;388}
% State (Depth:4) {36;87;89;388}
% State (Depth:3) {51;72;89;388}
% State (Depth:2) {17;51;144;388}
% State (Depth:1) {34;34;144;388}
% State (Depth:0) {0;68;144;388}

% --------------

% State (Depth:7) {9;101;158;332}
% State (Depth:6) {9;57;202;332}
% State (Depth:5) {18;48;202;332}
% State (Depth:4) {36;48;202;314}
% State (Depth:3) {12;72;202;314}
% State (Depth:2) {24;72;202;302}
% State (Depth:1) {48;48;202;302}
% State (Depth:0) {0;96;202;302}

% --------------

% State (Depth:7) {18;57;191;334}
% State (Depth:6) {18;114;134;334}
% State (Depth:5) {18;134;220;228}
% State (Depth:4) {8;18;134;440}
% State (Depth:3) {8;36;116;440}
% State (Depth:2) {8;72;80;440}
% State (Depth:1) {16;72;72;440}
% State (Depth:0) {0;16;144;440}

% --------------

% State (Depth:7) {18;106;111;365}
% State (Depth:6) {36;88;111;365}
% State (Depth:5) {72;75;88;365}
% State (Depth:4) {13;72;150;365}
% State (Depth:3) {13;78;144;365}
% State (Depth:2) {26;78;131;365}
% State (Depth:1) {52;52;131;365}
% State (Depth:0) {0;104;131;365}

% --------------

% State (Depth:7) {13;134;170;283}
% State (Depth:6) {13;36;268;283}
% State (Depth:5) {13;15;36;536}
% State (Depth:4) {15;26;36;523}
% State (Depth:3) {26;30;36;508}
% State (Depth:2) {10;30;52;508}
% State (Depth:1) {20;20;52;508}
% State (Depth:0) {0;40;52;508}

% --------------

% State (Depth:7) {2;41;157;400}
% State (Depth:6) {4;41;157;398}
% State (Depth:5) {8;41;153;398}
% State (Depth:4) {16;41;153;390}
% State (Depth:3) {32;41;137;390}
% State (Depth:2) {41;64;105;390}
% State (Depth:1) {41;41;128;390}
% State (Depth:0) {0;82;128;390}

% --------------

% State (Depth:7) {17;82;183;318}
% State (Depth:6) {34;82;166;318}
% State (Depth:5) {48;68;166;318}
% State (Depth:4) {20;96;166;318}
% State (Depth:3) {20;96;152;332}
% State (Depth:2) {40;76;152;332}
% State (Depth:1) {40;152;152;256}
% State (Depth:0) {0;40;256;304}

% --------------

% State (Depth:7) {1;99;181;319}
% State (Depth:6) {1;82;198;319}
% State (Depth:5) {2;82;198;318}
% State (Depth:4) {2;82;120;396}
% State (Depth:3) {4;80;120;396}
% State (Depth:2) {4;80;240;276}
% State (Depth:1) {4;160;160;276}
% State (Depth:0) {0;4;276;320}

% --------------

% State (Depth:7) {7;37;239;317}
% State (Depth:6) {7;37;78;478}
% State (Depth:5) {7;37;156;400}
% State (Depth:4) {7;74;156;363}
% State (Depth:3) {7;148;156;289}
% State (Depth:2) {7;8;289;296}
% State (Depth:1) {8;14;289;289}
% State (Depth:0) {0;8;14;578}

% --------------

% State (Depth:7) {5;67;143;385}
% State (Depth:6) {5;134;143;318}
% State (Depth:5) {5;143;184;268}
% State (Depth:4) {5;41;268;286}
% State (Depth:3) {5;18;41;536}
% State (Depth:2) {5;36;41;518}
% State (Depth:1) {5;5;72;518}
% State (Depth:0) {0;10;72;518}

% --------------

% State (Depth:7) {9;59;197;335}
% State (Depth:6) {9;118;197;276}
% State (Depth:5) {18;118;188;276}
% State (Depth:4) {18;158;188;236}
% State (Depth:3) {18;30;236;316}
% State (Depth:2) {12;36;236;316}
% State (Depth:1) {24;24;236;316}
% State (Depth:0) {0;48;236;316}

% --------------

% State (Depth:7) {5;122;151;322}
% State (Depth:6) {5;122;171;302}
% State (Depth:5) {5;122;131;342}
% State (Depth:4) {5;122;211;262}
% State (Depth:3) {5;89;244;262}
% State (Depth:2) {5;173;178;244}
% State (Depth:1) {5;5;244;346}
% State (Depth:0) {0;10;244;346}

% --------------

% State (Depth:7) {13;85;107;395}
% State (Depth:6) {13;107;170;310}
% State (Depth:5) {13;63;214;310}
% State (Depth:4) {13;63;96;428}
% State (Depth:3) {13;96;126;365}
% State (Depth:2) {13;96;239;252}
% State (Depth:1) {13;13;96;478}
% State (Depth:0) {0;26;96;478}

% --------------

% State (Depth:7) {67;71;129;333}
% State (Depth:6) {4;129;134;333}
% State (Depth:5) {4;129;199;268}
% State (Depth:4) {4;139;199;258}
% State (Depth:3) {4;60;258;278}
% State (Depth:2) {4;20;60;516}
% State (Depth:1) {4;40;40;516}
% State (Depth:0) {0;4;80;516}

% --------------

% State (Depth:7) {35;71;197;297}
% State (Depth:6) {35;126;142;297}
% State (Depth:5) {35;142;171;252}
% State (Depth:4) {29;35;252;284}
% State (Depth:3) {29;32;35;504}
% State (Depth:2) {29;35;64;472}
% State (Depth:1) {29;29;70;472}
% State (Depth:0) {0;58;70;472}

% --------------

% State (Depth:7) {1;99;219;281}
% State (Depth:6) {1;120;198;281}
% State (Depth:5) {2;119;198;281}
% State (Depth:4) {2;79;238;281}
% State (Depth:3) {2;158;159;281}
% State (Depth:2) {1;2;281;316}
% State (Depth:1) {2;2;281;315}
% State (Depth:0) {0;4;281;315}

% --------------

% State (Depth:7) {23;89;131;357}
% State (Depth:6) {46;89;108;357}
% State (Depth:5) {89;92;108;311}
% State (Depth:4) {19;92;178;311}
% State (Depth:3) {19;92;133;356}
% State (Depth:2) {38;92;114;356}
% State (Depth:1) {76;76;92;356}
% State (Depth:0) {0;92;152;356}

% --------------

% State (Depth:7) {3;35;197;365}
% State (Depth:6) {3;35;168;394}
% State (Depth:5) {6;35;168;391}
% State (Depth:4) {12;29;168;391}
% State (Depth:3) {24;29;168;379}
% State (Depth:2) {29;48;144;379}
% State (Depth:1) {29;96;96;379}
% State (Depth:0) {0;29;192;379}

% --------------

% State (Depth:7) {13;55;183;349}
% State (Depth:6) {26;55;170;349}
% State (Depth:5) {26;55;179;340}
% State (Depth:4) {29;52;179;340}
% State (Depth:3) {29;104;179;288}
% State (Depth:2) {29;179;184;208}
% State (Depth:1) {29;29;184;358}
% State (Depth:0) {0;58;184;358}

% --------------

% State (Depth:7) {13;107;157;323}
% State (Depth:6) {13;107;166;314}
% State (Depth:5) {13;107;148;332}
% State (Depth:4) {26;107;135;332}
% State (Depth:3) {26;135;214;225}
% State (Depth:2) {26;90;214;270}
% State (Depth:1) {26;180;180;214}
% State (Depth:0) {0;26;214;360}

% --------------

% State (Depth:7) {3;73;202;322}
% State (Depth:6) {6;73;202;319}
% State (Depth:5) {6;73;117;404}
% State (Depth:4) {12;73;111;404}
% State (Depth:3) {24;61;111;404}
% State (Depth:2) {37;48;111;404}
% State (Depth:1) {48;74;74;404}
% State (Depth:0) {0;48;148;404}

% --------------

% State (Depth:7) {13;89;207;291}
% State (Depth:6) {26;89;194;291}
% State (Depth:5) {26;178;194;202}
% State (Depth:4) {52;176;178;194}
% State (Depth:3) {104;124;178;194}
% State (Depth:2) {54;104;194;248}
% State (Depth:1) {54;54;104;388}
% State (Depth:0) {0;104;108;388}

% --------------

% State (Depth:7) {27;46;53;474}
% State (Depth:6) {19;53;54;474}
% State (Depth:5) {35;38;53;474}
% State (Depth:4) {18;38;70;474}
% State (Depth:3) {36;38;70;456}
% State (Depth:2) {34;38;72;456}
% State (Depth:1) {38;38;68;456}
% State (Depth:0) {0;68;76;456}

% --------------

% State (Depth:7) {45;85;151;319}
% State (Depth:6) {45;151;170;234}
% State (Depth:5) {45;64;151;340}
% State (Depth:4) {64;90;106;340}
% State (Depth:3) {26;106;128;340}
% State (Depth:2) {26;106;212;256}
% State (Depth:1) {26;150;212;212}
% State (Depth:0) {0;26;150;424}

% --------------

% State (Depth:7) {13;31;183;373}
% State (Depth:6) {18;26;183;373}
% State (Depth:5) {26;36;165;373}
% State (Depth:4) {26;36;208;330}
% State (Depth:3) {36;52;208;304}
% State (Depth:2) {36;104;208;252}
% State (Depth:1) {36;148;208;208}
% State (Depth:0) {0;36;148;416}

% --------------

% State (Depth:7) {53;71;236;240}
% State (Depth:6) {53;142;165;240}
% State (Depth:5) {23;53;240;284}
% State (Depth:4) {23;44;53;480}
% State (Depth:3) {21;46;53;480}
% State (Depth:2) {7;21;92;480}
% State (Depth:1) {14;14;92;480}
% State (Depth:0) {0;28;92;480}

% --------------

% State (Depth:7) {1;89;224;286}
% State (Depth:6) {2;89;224;285}
% State (Depth:5) {4;89;224;283}
% State (Depth:4) {8;89;220;283}
% State (Depth:3) {8;178;194;220}
% State (Depth:2) {16;178;194;212}
% State (Depth:1) {16;16;212;356}
% State (Depth:0) {0;32;212;356}

% --------------

% State (Depth:7) {19;74;181;326}
% State (Depth:6) {19;74;145;362}
% State (Depth:5) {19;145;148;288}
% State (Depth:4) {3;19;288;290}
% State (Depth:3) {6;16;288;290}
% State (Depth:2) {2;6;16;576}
% State (Depth:1) {4;4;16;576}
% State (Depth:0) {0;8;16;576}

% --------------

% State (Depth:7) {29;107;151;313}
% State (Depth:6) {29;107;162;302}
% State (Depth:5) {29;162;195;214}
% State (Depth:4) {29;52;195;324}
% State (Depth:3) {52;58;166;324}
% State (Depth:2) {52;108;116;324}
% State (Depth:1) {52;116;216;216}
% State (Depth:0) {0;52;116;432}

% --------------

% State (Depth:7) {1;119;169;311}
% State (Depth:6) {2;119;169;310}
% State (Depth:5) {4;119;169;308}
% State (Depth:4) {4;119;139;338}
% State (Depth:3) {8;119;135;338}
% State (Depth:2) {8;16;238;338}
% State (Depth:1) {16;16;238;330}
% State (Depth:0) {0;32;238;330}

% --------------

% State (Depth:7) {9;92;158;341}
% State (Depth:6) {9;158;184;249}
% State (Depth:5) {9;91;184;316}
% State (Depth:4) {9;91;132;368}
% State (Depth:3) {9;91;236;264}
% State (Depth:2) {9;173;182;236}
% State (Depth:1) {9;9;236;346}
% State (Depth:0) {0;18;236;346}

% --------------

% State (Depth:7) {3;53;199;345}
% State (Depth:6) {6;50;199;345}
% State (Depth:5) {12;50;199;339}
% State (Depth:4) {12;100;149;339}
% State (Depth:3) {24;100;137;339}
% State (Depth:2) {48;100;113;339}
% State (Depth:1) {48;100;226;226}
% State (Depth:0) {0;48;100;452}

% --------------

% State (Depth:7) {2;81;200;317}
% State (Depth:6) {2;81;117;400}
% State (Depth:5) {4;81;115;400}
% State (Depth:4) {4;115;162;319}
% State (Depth:3) {4;162;204;230}
% State (Depth:2) {4;68;204;324}
% State (Depth:1) {4;136;136;324}
% State (Depth:0) {0;4;272;324}

% --------------

% State (Depth:7) {9;79;127;385}
% State (Depth:6) {9;127;158;306}
% State (Depth:5) {18;127;149;306}
% State (Depth:4) {18;149;179;254}
% State (Depth:3) {36;149;161;254}
% State (Depth:2) {12;36;254;298}
% State (Depth:1) {24;24;254;298}
% State (Depth:0) {0;48;254;298}

% --------------

% State (Depth:7) {11;102;200;287}
% State (Depth:6) {11;185;200;204}
% State (Depth:5) {22;174;200;204}
% State (Depth:4) {22;26;204;348}
% State (Depth:3) {22;52;178;348}
% State (Depth:2) {22;104;126;348}
% State (Depth:1) {44;104;104;348}
% State (Depth:0) {0;44;208;348}

% --------------

% State (Depth:7) {1;156;200;243}
% State (Depth:6) {1;43;156;400}
% State (Depth:5) {2;42;156;400}
% State (Depth:4) {4;40;156;400}
% State (Depth:3) {8;40;152;400}
% State (Depth:2) {16;32;152;400}
% State (Depth:1) {32;32;136;400}
% State (Depth:0) {0;64;136;400}

% --------------

% State (Depth:7) {43;59;141;357}
% State (Depth:6) {43;59;216;282}
% State (Depth:5) {59;86;173;282}
% State (Depth:4) {27;118;173;282}
% State (Depth:3) {27;109;118;346}
% State (Depth:2) {9;27;218;346}
% State (Depth:1) {18;18;218;346}
% State (Depth:0) {0;36;218;346}

% --------------

% State (Depth:7) {5;124;142;329}
% State (Depth:6) {10;124;137;329}
% State (Depth:5) {10;13;248;329}
% State (Depth:4) {13;20;238;329}
% State (Depth:3) {13;20;91;476}
% State (Depth:2) {20;26;78;476}
% State (Depth:1) {20;52;52;476}
% State (Depth:0) {0;20;104;476}

% --------------

% State (Depth:7) {1;99;161;339}
% State (Depth:6) {2;99;161;338}
% State (Depth:5) {4;99;159;338}
% State (Depth:4) {4;60;198;338}
% State (Depth:3) {8;56;198;338}
% State (Depth:2) {16;48;198;338}
% State (Depth:1) {32;32;198;338}
% State (Depth:0) {0;64;198;338}

% --------------

% State (Depth:7) {7;78;122;393}
% State (Depth:6) {14;78;115;393}
% State (Depth:5) {28;78;101;393}
% State (Depth:4) {56;73;78;393}
% State (Depth:3) {73;78;112;337}
% State (Depth:2) {39;78;146;337}
% State (Depth:1) {78;78;107;337}
% State (Depth:0) {0;107;156;337}

% --------------

% State (Depth:7) {8;15;197;380}
% State (Depth:6) {15;16;189;380}
% State (Depth:5) {15;32;189;364}
% State (Depth:4) {15;64;189;332}
% State (Depth:3) {15;64;143;378}
% State (Depth:2) {30;64;128;378}
% State (Depth:1) {30;128;128;314}
% State (Depth:0) {0;30;256;314}

% --------------

% State (Depth:7) {6;83;200;311}
% State (Depth:6) {12;83;200;305}
% State (Depth:5) {12;166;200;222}
% State (Depth:4) {12;56;200;332}
% State (Depth:3) {12;56;132;400}
% State (Depth:2) {24;44;132;400}
% State (Depth:1) {24;88;88;400}
% State (Depth:0) {0;24;176;400}

% --------------

% State (Depth:7) {3;113;198;286}
% State (Depth:6) {6;113;198;283}
% State (Depth:5) {6;85;226;283}
% State (Depth:4) {12;79;226;283}
% State (Depth:3) {12;158;204;226}
% State (Depth:2) {12;68;204;316}
% State (Depth:1) {12;136;136;316}
% State (Depth:0) {0;12;272;316}

% --------------

% State (Depth:7) {17;65;183;335}
% State (Depth:6) {34;65;166;335}
% State (Depth:5) {31;68;166;335}
% State (Depth:4) {62;68;135;335}
% State (Depth:3) {62;67;136;335}
% State (Depth:2) {62;134;136;268}
% State (Depth:1) {2;62;268;268}
% State (Depth:0) {0;2;62;536}

% --------------

% State (Depth:7) {35;57;251;257}
% State (Depth:6) {6;35;57;502}
% State (Depth:5) {12;35;57;496}
% State (Depth:4) {24;35;57;484}
% State (Depth:3) {35;48;57;460}
% State (Depth:2) {13;57;70;460}
% State (Depth:1) {13;13;114;460}
% State (Depth:0) {0;26;114;460}

% --------------

% State (Depth:7) {17;58;166;359}
% State (Depth:6) {17;108;116;359}
% State (Depth:5) {34;91;116;359}
% State (Depth:4) {34;91;232;243}
% State (Depth:3) {68;91;209;232}
% State (Depth:2) {23;68;91;418}
% State (Depth:1) {23;23;136;418}
% State (Depth:0) {0;46;136;418}

% --------------

% State (Depth:7) {5;31;179;385}
% State (Depth:6) {10;31;179;380}
% State (Depth:5) {10;62;148;380}
% State (Depth:4) {10;86;124;380}
% State (Depth:3) {10;86;248;256}
% State (Depth:2) {10;162;172;256}
% State (Depth:1) {10;10;256;324}
% State (Depth:0) {0;20;256;324}

% --------------

% State (Depth:7) {19;41;181;359}
% State (Depth:6) {22;38;181;359}
% State (Depth:5) {38;44;159;359}
% State (Depth:4) {38;88;159;315}
% State (Depth:3) {38;71;176;315}
% State (Depth:2) {38;105;142;315}
% State (Depth:1) {38;142;210;210}
% State (Depth:0) {0;38;142;420}

% --------------

% State (Depth:7) {3;47;201;349}
% State (Depth:6) {6;47;201;346}
% State (Depth:5) {6;47;145;402}
% State (Depth:4) {12;47;145;396}
% State (Depth:3) {12;94;98;396}
% State (Depth:2) {4;12;188;396}
% State (Depth:1) {8;8;188;396}
% State (Depth:0) {0;16;188;396}

% --------------

% State (Depth:7) {11;153;189;247}
% State (Depth:6) {11;36;247;306}
% State (Depth:5) {11;36;59;494}
% State (Depth:4) {11;59;72;458}
% State (Depth:3) {22;48;72;458}
% State (Depth:2) {22;48;144;386}
% State (Depth:1) {22;96;96;386}
% State (Depth:0) {0;22;192;386}

% --------------

% State (Depth:7) {1;89;200;310}
% State (Depth:6) {2;89;199;310}
% State (Depth:5) {4;89;199;308}
% State (Depth:4) {8;89;195;308}
% State (Depth:3) {8;89;113;390}
% State (Depth:2) {16;89;105;390}
% State (Depth:1) {32;89;89;390}
% State (Depth:0) {0;32;178;390}

% --------------

% State (Depth:7) {29;34;142;395}
% State (Depth:6) {5;58;142;395}
% State (Depth:5) {5;116;142;337}
% State (Depth:4) {5;116;195;284}
% State (Depth:3) {5;79;232;284}
% State (Depth:2) {5;153;158;284}
% State (Depth:1) {5;5;284;306}
% State (Depth:0) {0;10;284;306}

% --------------

% State (Depth:7) {3;77;197;323}
% State (Depth:6) {6;77;194;323}
% State (Depth:5) {6;77;129;388}
% State (Depth:4) {12;77;123;388}
% State (Depth:3) {24;65;123;388}
% State (Depth:2) {41;48;123;388}
% State (Depth:1) {48;82;82;388}
% State (Depth:0) {0;48;164;388}

% --------------

% State (Depth:7) {43;49;59;449}
% State (Depth:6) {16;49;86;449}
% State (Depth:5) {16;49;172;363}
% State (Depth:4) {32;33;172;363}
% State (Depth:3) {32;66;172;330}
% State (Depth:2) {32;132;172;264}
% State (Depth:1) {32;40;264;264}
% State (Depth:0) {0;32;40;528}

% --------------

% State (Depth:7) {67;167;181;185}
% State (Depth:6) {100;134;181;185}
% State (Depth:5) {47;100;185;268}
% State (Depth:4) {94;100;138;268}
% State (Depth:3) {44;100;188;268}
% State (Depth:2) {44;88;200;268}
% State (Depth:1) {88;88;200;224}
% State (Depth:0) {0;176;200;224}

% --------------

% State (Depth:7) {53;71;134;342}
% State (Depth:6) {53;134;142;271}
% State (Depth:5) {106;134;142;218}
% State (Depth:4) {112;134;142;212}
% State (Depth:3) {8;112;212;268}
% State (Depth:2) {8;56;112;424}
% State (Depth:1) {8;112;112;368}
% State (Depth:0) {0;8;224;368}

% --------------

% State (Depth:7) {1;77;177;345}
% State (Depth:6) {1;77;168;354}
% State (Depth:5) {1;77;186;336}
% State (Depth:4) {1;77;150;372}
% State (Depth:3) {2;77;150;371}
% State (Depth:2) {4;75;150;371}
% State (Depth:1) {4;150;150;296}
% State (Depth:0) {0;4;296;300}

% --------------

% State (Depth:7) {3;74;197;326}
% State (Depth:6) {6;74;197;323}
% State (Depth:5) {12;68;197;323}
% State (Depth:4) {24;56;197;323}
% State (Depth:3) {32;48;197;323}
% State (Depth:2) {32;96;149;323}
% State (Depth:1) {64;64;149;323}
% State (Depth:0) {0;128;149;323}

% --------------

% State (Depth:7) {70;95;101;334}
% State (Depth:6) {25;101;140;334}
% State (Depth:5) {50;101;115;334}
% State (Depth:4) {50;101;219;230}
% State (Depth:3) {50;129;202;219}
% State (Depth:2) {50;73;219;258}
% State (Depth:1) {50;146;146;258}
% State (Depth:0) {0;50;258;292}

% --------------

% State (Depth:7) {28;47;124;401}
% State (Depth:6) {28;77;94;401}
% State (Depth:5) {28;77;188;307}
% State (Depth:4) {28;77;119;376}
% State (Depth:3) {56;77;91;376}
% State (Depth:2) {21;91;112;376}
% State (Depth:1) {42;91;91;376}
% State (Depth:0) {0;42;182;376}

% --------------

% State (Depth:7) {1;101;177;321}
% State (Depth:6) {1;101;144;354}
% State (Depth:5) {2;100;144;354}
% State (Depth:4) {4;100;144;352}
% State (Depth:3) {8;96;144;352}
% State (Depth:2) {8;96;208;288}
% State (Depth:1) {8;192;192;208}
% State (Depth:0) {0;8;208;384}

% --------------

% State (Depth:7) {65;90;146;299}
% State (Depth:6) {65;146;180;209}
% State (Depth:5) {130;144;146;180}
% State (Depth:4) {36;130;146;288}
% State (Depth:3) {72;94;146;288}
% State (Depth:2) {22;144;146;288}
% State (Depth:1) {2;22;288;288}
% State (Depth:0) {0;2;22;576}

% --------------

% State (Depth:7) {7;120;146;327}
% State (Depth:6) {14;113;146;327}
% State (Depth:5) {14;146;214;226}
% State (Depth:4) {28;132;214;226}
% State (Depth:3) {56;132;198;214}
% State (Depth:2) {16;56;132;396}
% State (Depth:1) {16;56;264;264}
% State (Depth:0) {0;16;56;528}

% --------------

% State (Depth:7) {9;149;197;245}
% State (Depth:6) {18;149;197;236}
% State (Depth:5) {18;39;149;394}
% State (Depth:4) {18;78;110;394}
% State (Depth:3) {18;78;220;284}
% State (Depth:2) {18;142;156;284}
% State (Depth:1) {14;18;284;284}
% State (Depth:0) {0;14;18;568}

% --------------

% State (Depth:7) {1;79;199;321}
% State (Depth:6) {2;79;199;320}
% State (Depth:5) {4;79;197;320}
% State (Depth:4) {8;79;197;316}
% State (Depth:3) {16;79;189;316}
% State (Depth:2) {32;63;189;316}
% State (Depth:1) {32;126;126;316}
% State (Depth:0) {0;32;252;316}

% --------------

% State (Depth:7) {45;85;168;302}
% State (Depth:6) {45;85;134;336}
% State (Depth:5) {40;90;134;336}
% State (Depth:4) {80;90;134;296}
% State (Depth:3) {54;90;160;296}
% State (Depth:2) {36;108;160;296}
% State (Depth:1) {72;72;160;296}
% State (Depth:0) {0;144;160;296}

% --------------

% State (Depth:7) {4;43;205;348}
% State (Depth:6) {4;86;205;305}
% State (Depth:5) {4;172;205;219}
% State (Depth:4) {4;33;219;344}
% State (Depth:3) {4;66;186;344}
% State (Depth:2) {8;62;186;344}
% State (Depth:1) {8;124;124;344}
% State (Depth:0) {0;8;248;344}

% --------------

% State (Depth:7) {1;121;158;320}
% State (Depth:6) {2;121;158;319}
% State (Depth:5) {2;158;198;242}
% State (Depth:4) {4;158;198;240}
% State (Depth:3) {4;40;240;316}
% State (Depth:2) {4;80;240;276}
% State (Depth:1) {4;160;160;276}
% State (Depth:0) {0;4;276;320}

% --------------

% State (Depth:7) {7;97;183;313}
% State (Depth:6) {7;86;194;313}
% State (Depth:5) {7;172;194;227}
% State (Depth:4) {14;165;194;227}
% State (Depth:3) {14;33;165;388}
% State (Depth:2) {14;66;132;388}
% State (Depth:1) {14;132;132;322}
% State (Depth:0) {0;14;264;322}

% --------------

% State (Depth:7) {61;113;127;299}
% State (Depth:6) {61;127;186;226}
% State (Depth:5) {122;125;127;226}
% State (Depth:4) {104;125;127;244}
% State (Depth:3) {2;104;244;250}
% State (Depth:2) {2;6;104;488}
% State (Depth:1) {4;4;104;488}
% State (Depth:0) {0;8;104;488}

% --------------

% State (Depth:7) {15;37;205;343}
% State (Depth:6) {15;74;205;306}
% State (Depth:5) {30;74;190;306}
% State (Depth:4) {44;60;190;306}
% State (Depth:3) {16;88;190;306}
% State (Depth:2) {16;102;176;306}
% State (Depth:1) {16;176;204;204}
% State (Depth:0) {0;16;176;408}

% --------------

% State (Depth:7) {9;96;106;389}
% State (Depth:6) {9;96;212;283}
% State (Depth:5) {9;71;96;424}
% State (Depth:4) {9;25;142;424}
% State (Depth:3) {18;25;133;424}
% State (Depth:2) {18;25;266;291}
% State (Depth:1) {18;50;266;266}
% State (Depth:0) {0;18;50;532}

% --------------

% State (Depth:7) {57;71;81;391}
% State (Depth:6) {57;71;162;310}
% State (Depth:5) {57;91;142;310}
% State (Depth:4) {91;114;142;253}
% State (Depth:3) {23;142;182;253}
% State (Depth:2) {23;71;142;364}
% State (Depth:1) {23;142;142;293}
% State (Depth:0) {0;23;284;293}

% --------------

% State (Depth:7) {42;74;181;303}
% State (Depth:6) {42;148;181;229}
% State (Depth:5) {42;81;181;296}
% State (Depth:4) {39;84;181;296}
% State (Depth:3) {39;168;181;212}
% State (Depth:2) {13;39;212;336}
% State (Depth:1) {26;26;212;336}
% State (Depth:0) {0;52;212;336}

% --------------

% State (Depth:7) {13;41;159;387}
% State (Depth:6) {13;82;118;387}
% State (Depth:5) {26;82;105;387}
% State (Depth:4) {26;82;210;282}
% State (Depth:3) {26;128;164;282}
% State (Depth:2) {26;36;256;282}
% State (Depth:1) {26;26;36;512}
% State (Depth:0) {0;36;52;512}

% --------------

% State (Depth:7) {1;79;197;323}
% State (Depth:6) {1;79;126;394}
% State (Depth:5) {2;79;126;393}
% State (Depth:4) {2;79;252;267}
% State (Depth:3) {2;158;188;252}
% State (Depth:2) {2;94;188;316}
% State (Depth:1) {2;188;188;222}
% State (Depth:0) {0;2;222;376}

% --------------

% State (Depth:7) {9;98;155;338}
% State (Depth:6) {9;155;196;240}
% State (Depth:5) {9;85;196;310}
% State (Depth:4) {9;111;170;310}
% State (Depth:3) {18;102;170;310}
% State (Depth:2) {18;68;204;310}
% State (Depth:1) {18;136;136;310}
% State (Depth:0) {0;18;272;310}

% --------------

% State (Depth:7) {3;112;198;287}
% State (Depth:6) {6;112;198;284}
% State (Depth:5) {12;106;198;284}
% State (Depth:4) {12;178;198;212}
% State (Depth:3) {12;20;212;356}
% State (Depth:2) {8;24;212;356}
% State (Depth:1) {16;16;212;356}
% State (Depth:0) {0;32;212;356}

% --------------

% State (Depth:7) {1;163;200;236}
% State (Depth:6) {2;163;200;235}
% State (Depth:5) {2;35;163;400}
% State (Depth:4) {2;70;128;400}
% State (Depth:3) {4;68;128;400}
% State (Depth:2) {8;64;128;400}
% State (Depth:1) {8;128;128;336}
% State (Depth:0) {0;8;256;336}

% --------------

% State (Depth:7) {17;83;163;337}
% State (Depth:6) {34;83;146;337}
% State (Depth:5) {68;83;112;337}
% State (Depth:4) {44;83;136;337}
% State (Depth:3) {83;88;92;337}
% State (Depth:2) {83;92;176;249}
% State (Depth:1) {92;166;166;176}
% State (Depth:0) {0;92;176;332}

% --------------

% State (Depth:7) {6;19;219;356}
% State (Depth:6) {6;38;219;337}
% State (Depth:5) {12;32;219;337}
% State (Depth:4) {24;32;207;337}
% State (Depth:3) {24;32;130;414}
% State (Depth:2) {24;32;260;284}
% State (Depth:1) {24;24;32;520}
% State (Depth:0) {0;32;48;520}

% --------------

% State (Depth:7) {25;173;185;217}
% State (Depth:6) {50;160;173;217}
% State (Depth:5) {100;123;160;217}
% State (Depth:4) {60;123;200;217}
% State (Depth:3) {63;120;200;217}
% State (Depth:2) {63;80;217;240}
% State (Depth:1) {63;160;160;217}
% State (Depth:0) {0;63;217;320}

\begin{lstcs}
--------------

State (Depth:7) {7;122;193;278}
State (Depth:6) {7;85;122;386}
State (Depth:5) {14;85;115;386}
State (Depth:4) {28;85;115;372}
State (Depth:3) {56;57;115;372}
State (Depth:2) {56;58;114;372}
State (Depth:1) {56;56;116;372}
State (Depth:0) {0;112;116;372}

--------------

State (Depth:7) {6;31;202;361}
State (Depth:6) {12;31;196;361}
State (Depth:5) {12;31;165;392}
State (Depth:4) {24;31;165;380}
State (Depth:3) {24;62;134;380}
State (Depth:2) {24;72;124;380}
State (Depth:1) {48;48;124;380}
State (Depth:0) {0;96;124;380}

--------------

State (Depth:7) {1;89;221;289}
State (Depth:6) {2;88;221;289}
State (Depth:5) {4;88;221;287}
State (Depth:4) {8;88;221;283}
State (Depth:3) {16;80;221;283}
State (Depth:2) {32;64;221;283}
State (Depth:1) {64;64;189;283}
State (Depth:0) {0;128;189;283}

--------------

State (Depth:7) {5;31;193;371}
State (Depth:6) {5;31;178;386}
State (Depth:5) {10;31;178;381}
State (Depth:4) {10;62;147;381}
State (Depth:3) {20;52;147;381}
State (Depth:2) {40;52;127;381}
State (Depth:1) {40;52;254;254}
State (Depth:0) {0;40;52;508}

--------------


FERTIG!
Man benötigt 8 Telepaartie-Schritte
Die Berechnung dauerte 5.33s.
\end{lstcs}

Für \(n = 5000\) ist die Ausgabe für die LLL:
\begin{lstcs}
Telepaartie.CLI -c 3 -e 5000 -v

Starting iteration 15

--------------

State (Depth:14) {125;1558;3317}
State (Depth:13) {250;1558;3192}
State (Depth:12) {250;1634;3116}
State (Depth:11) {500;1384;3116}
State (Depth:10) {500;1732;2768}
State (Depth:9) {1000;1232;2768}
State (Depth:8) {232;2000;2768}
State (Depth:7) {464;1768;2768}
State (Depth:6) {928;1304;2768}
State (Depth:5) {1304;1840;1856}
State (Depth:4) {552;1840;2608}
State (Depth:3) {1104;1840;2056}
State (Depth:2) {736;2056;2208}
State (Depth:1) {1472;1472;2056}
State (Depth:0) {0;2056;2944}

--------------

State (Depth:14) {125;1849;3026}
State (Depth:13) {250;1849;2901}
State (Depth:12) {500;1599;2901}
State (Depth:11) {1000;1099;2901}
State (Depth:10) {1000;1802;2198}
State (Depth:9) {396;1000;3604}
State (Depth:8) {792;1000;3208}
State (Depth:7) {208;1584;3208}
State (Depth:6) {416;1376;3208}
State (Depth:5) {832;960;3208}
State (Depth:4) {128;1664;3208}
State (Depth:3) {256;1536;3208}
State (Depth:2) {512;1536;2952}
State (Depth:1) {1024;1024;2952}
State (Depth:0) {0;2048;2952}

--------------
\end{lstcs}

- Einige Ergebnisse der Kürze halber ausgelassen -

% State (Depth:14) {125;1724;3151}
% State (Depth:13) {125;1427;3448}
% State (Depth:12) {250;1302;3448}
% State (Depth:11) {500;1052;3448}
% State (Depth:10) {500;2104;2396}
% State (Depth:9) {1000;1604;2396}
% State (Depth:8) {792;1000;3208}
% State (Depth:7) {208;1584;3208}
% State (Depth:6) {416;1376;3208}
% State (Depth:5) {832;960;3208}
% State (Depth:4) {128;1664;3208}
% State (Depth:3) {256;1536;3208}
% State (Depth:2) {512;1536;2952}
% State (Depth:1) {1024;1024;2952}
% State (Depth:0) {0;2048;2952}

% --------------

% State (Depth:14) {125;1901;2974}
% State (Depth:13) {125;1073;3802}
% State (Depth:12) {250;1073;3677}
% State (Depth:11) {250;2146;2604}
% State (Depth:10) {500;2146;2354}
% State (Depth:9) {208;500;4292}
% State (Depth:8) {292;416;4292}
% State (Depth:7) {292;832;3876}
% State (Depth:6) {292;1664;3044}
% State (Depth:5) {292;1380;3328}
% State (Depth:4) {292;1948;2760}
% State (Depth:3) {584;1656;2760}
% State (Depth:2) {584;1104;3312}
% State (Depth:1) {584;2208;2208}
% State (Depth:0) {0;584;4416}

% --------------

% State (Depth:14) {125;1776;3099}
% State (Depth:13) {125;1323;3552}
% State (Depth:12) {125;2229;2646}
% State (Depth:11) {125;417;4458}
% State (Depth:10) {250;292;4458}
% State (Depth:9) {292;500;4208}
% State (Depth:8) {208;584;4208}
% State (Depth:7) {208;1168;3624}
% State (Depth:6) {416;960;3624}
% State (Depth:5) {832;960;3208}
% State (Depth:4) {128;1664;3208}
% State (Depth:3) {256;1536;3208}
% State (Depth:2) {512;1536;2952}
% State (Depth:1) {1024;1024;2952}
% State (Depth:0) {0;2048;2952}

\begin{lstcs}
--------------

State (Depth:14) {125;1658;3217}
State (Depth:13) {250;1658;3092}
State (Depth:12) {500;1408;3092}
State (Depth:11) {500;1684;2816}
State (Depth:10) {1000;1184;2816}
State (Depth:9) {1184;1816;2000}
State (Depth:8) {816;1816;2368}
State (Depth:7) {552;816;3632}
State (Depth:6) {816;1104;3080}
State (Depth:5) {288;1632;3080}
State (Depth:4) {576;1344;3080}
State (Depth:3) {768;1152;3080}
State (Depth:2) {768;1928;2304}
State (Depth:1) {1536;1536;1928}
State (Depth:0) {0;1928;3072}

--------------

State (Depth:14) {125;1403;3472}
State (Depth:13) {250;1403;3347}
State (Depth:12) {500;1153;3347}
State (Depth:11) {500;2194;2306}
State (Depth:10) {112;500;4388}
State (Depth:9) {112;1000;3888}
State (Depth:8) {112;2000;2888}
State (Depth:7) {224;1888;2888}
State (Depth:6) {448;1888;2664}
State (Depth:5) {896;1888;2216}
State (Depth:4) {1320;1792;1888}
State (Depth:3) {472;1888;2640}
State (Depth:2) {944;1888;2168}
State (Depth:1) {1224;1888;1888}
State (Depth:0) {0;1224;3776}

--------------


FERTIG!
Man benötigt 15 Telepaartie-Schritte
Die Berechnung dauerte 7.289s.
\end{lstcs}

\section{Quellcode}

\begin{lstcs}
using System;
using System.Collections.Generic;
using System.Linq;

public static class Telepaartie
{
    private const string _separator = "--------------";

    public static int L(
        IEnumerable<int> goalBuckets,
        Action<string>? writeLine = null)   //Zum finden der minimalen Anzahl an Operationen für einen Zustand
    {
        if (goalBuckets == null) throw new ArgumentNullException(nameof(goalBuckets));

        var goal = new State(goalBuckets);

        var numberOfCups = goalBuckets.Count();
        var numberOfItems = goalBuckets.Sum();

        return LLLCore(numberOfCups, numberOfItems, goal, writeLine);
    }

    public static int LLL(
        int numberOfCups = 3,
        int numberOfItems = 15,
        Action<string>? writeLine = null)   //Zum finden der maximalen Anzahl der minimalen Anzahlen an Operationen für eine Anzahl
    {
        return LLLCore(numberOfCups, numberOfItems, null, writeLine);
    }

    private static int LLLCore(
        int numberOfCups,
        int numberOfItems,
        State? goal,
        Action<string>? writeLine)
    {
        HashSet<State> lastGen = new HashSet<State>(
            // Alle Endzustände bilden die nullte Generation
            State.AllEndingStates(numberOfCups, numberOfItems)
            .Select(x => new State(x)));

        HashSet<State> allStates = new HashSet<State>(lastGen);

        for (int i = 0; ; i++)
        {
            writeLine?.Invoke($"\rStarting iteration {i + 1}");

            HashSet<State> nextGen = new HashSet<State>(lastGen
                // Aktiviere Parallelisierung mit PLINQ
                .AsParallel()
                // Ermittle alle Ursprungzustände
                .SelectMany(x => x.Origins()));

            // Entferne Zustände die schon in vorherigen Generationen vorhanden sind
            lastGen.ExceptWith(allStates);

            // Falls die Operationsanzahl für nur einen Zustand festgestellt werden soll
            if (goal != null)
            {
                // Wenn das Element in der neuen Generation vorhanden ist, gebe den Generationsindex, also die Anzahl zum lösen benötigter Telepaartien zurück
                if (nextGen.Contains(goal)) return i + 1;
            }
            // Wenn die neue Generation die leere Menge ist
            else if (nextGen.Count == 0)
            {
                // Output
                if (writeLine != null)
                {
                    writeLine(Environment.NewLine);
                    foreach (var oldestChild in lastGen)
                    {
                        writeLine(Environment.NewLine + _separator + Environment.NewLine + Environment.NewLine);

                        for (State? current = oldestChild; current != null; current = current.Parent)
                        {
                            writeLine(current.ToString() + Environment.NewLine);
                        }
                    }

                    writeLine(Environment.NewLine + _separator + Environment.NewLine + Environment.NewLine);
                }

                // Gebe den Generationsindex, also die Anzahl zum lösen benötigter Telepaartien zurück
                return i + 1;
            }

            // Zur Sammlung aller bisher entdeckten Zustände die jetzige Generation hinzufügen.
            allStates.UnionWith(nextGen);
            // Die letzte Generation durch die jetzige ersetzen, um die nächste korrekt ausrechnen zu lassen
            lastGen = nextGen;
        }
    }
}
\end{lstcs}
\begin{lstcs}
using System;
using System.Collections.Generic;
using System.Linq;

public class State : IEquatable<State>
{
    public int Depth => Parent == null ? 0 : (Parent.Depth + 1);
    
    public State? Parent { get; }
    
    public int[] Buckets { get; }
    
    private readonly int _hashCode;
    
    public State(IEnumerable<int> unsortedBuckets, State? parent = null)
    {
        if (unsortedBuckets.Any(x => x < 0)) throw new ArgumentException(nameof(unsortedBuckets));
    
        Buckets = unsortedBuckets.ToArray();
        Array.Sort(Buckets);
    
        Parent = parent;
        _hashCode = CalculateHashCode();
    }
    
    private State(int[] sortedBuckets, State? parent = null)
    {
        Buckets = sortedBuckets;
        Parent = parent;
        _hashCode = CalculateHashCode();
    }
    
    private State ReverseTeelepartie(int originalTarget, int originalSource)
    {
        int[] temp = new int[Buckets.Length];
        Buckets.CopyTo(temp, 0);
    
        temp[originalTarget] /= 2;
        temp[originalSource] += temp[originalTarget];
        Array.Sort(temp);
    
        return new State(temp, this);
    }
    
    public IEnumerable<State> Origins()
    {
        // Finden jeder Kombination
        for (int i = 0; i < Buckets.Length; i++)
        {
            for (int u = 0; u <= i; u++)
            {
                // Zulässige Werte rausfiltern
                if (Buckets[i] % 2 == 0 && Buckets[i] > 0)
                {
                    // und die bearbeitete Version zurückgeben
                    yield return ReverseTeelepartie(i, u);
                }
    
                if (Buckets[u] % 2 == 0 && Buckets[u] > 0)
                {
                    yield return ReverseTeelepartie(u, i);
                }
            }
        }
    }
    
    private int CalculateHashCode() =>
        Buckets.Aggregate(168560841, (x, y) => (x * -1521134295) + y);
    
    public static IEnumerable<List<int>> AllEndingStates(int numberOfCups, int numberOfItems)
    {
        foreach (var state in AllPossibleStates(numberOfCups - 1, numberOfItems, numberOfItems))
        {
            state.Add(0);
            yield return state;
        }
    }
    
    public static IEnumerable<List<int>> AllPossibleStates(int numberOfCups, int numberOfItems, int previousMax)
    {
        if (numberOfCups < 1) yield break;
        if (numberOfCups == 1) yield return new List<int> { numberOfItems };
    
        // Die Elementanzahl die mindestens dem aktuellen Behälter hinzugefügt werden muss
        int min = ((numberOfItems - 1) / numberOfCups) + 1;
    
        // Die Elementanzahl die maximal dem aktuellen Behälter hinzugefügt werden kann
        int max = Math.Min(previousMax + 1, numberOfItems);
    
        for (int i = min; i < max; i++)
        {
            // Finden aller Möglichen Kombinationen für den Rest der Biber und der Behälteranzahl -1
            foreach (var state in AllPossibleStates(numberOfCups - 1, numberOfItems - i, i))
            {
                state.Add(i);
                yield return state;
            }
        }
    }
}
\end{lstcs}
\end{document}

